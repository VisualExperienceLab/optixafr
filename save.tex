% ---------------------------------------------------------------------------
% Author guideline and sample document for EG publication using LaTeX2e input
% D.Fellner, v1.13, Jul 31, 2008

\documentclass{egpubl}
\usepackage{pg2016}

% --- for  Annual CONFERENCE
% \ConferenceSubmission % uncomment for Conference submission
% \ConferencePaper      % uncomment for (final) Conference Paper
% \STAR                 % uncomment for STAR contribution
% \Tutorial             % uncomment for Tutorial contribution
% \ShortPresentation    % uncomment for (final) Short Conference Presentation
%
% --- for  CGF Journal
% \JournalSubmission    % uncomment for submission to Computer Graphics Forum
% \JournalPaper         % uncomment for final version of Journal Paper
%
% --- for  CGF Journal: special issue
 \SpecialIssueSubmission    % uncomment for submission to Computer Graphics Forum, special issue
% \SpecialIssuePaper         % uncomment for final version of Journal Paper, special issue
%
% --- for  EG Workshop Proceedings
% \WsSubmission    % uncomment for submission to EG Workshop
% \WsPaper         % uncomment for final version of EG Workshop contribution
%
 \electronicVersion % can be used both for the printed and electronic version

% !! *please* don't change anything above
% !! unless you REALLY know what you are doing
% ------------------------------------------------------------------------

% for including postscript figures
% mind: package option 'draft' will replace PS figure by a filename within a frame
\ifpdf \usepackage[pdftex]{graphicx} \pdfcompresslevel=9
\else \usepackage[dvips]{graphicx} \fi

\PrintedOrElectronic

% prepare for electronic version of your document
\usepackage{t1enc,dfadobe}

\usepackage{egweblnk}
\usepackage{cite}

% For backwards compatibility to old LaTeX type font selection.
% Uncomment if your document adheres to LaTeX2e recommendations.
% \let\rm=\rmfamily    \let\sf=\sffamily    \let\tt=\ttfamily
% \let\it=\itshape     \let\sl=\slshape     \let\sc=\scshape
% \let\bf=\bfseries

% end of prologue

% ---------------------------------------------------------------------
% EG author guidelines plus sample file for EG publication using LaTeX2e input
% D.Fellner, v1.17, Sep 23, 2010


\title{An Real Time Implementation of Adaptive Frameless Rendering}

% for anonymous conference submission please enter your SUBMISSION ID
% instead of the author's name (and leave the affiliation blank) !!
\author[paperID]{paperID}

%\author[D. Fellner \& S. Behnke]
%       {D.\,W. Fellner\thanks{Chairman Eurographics Publications Board}$^{1,2}$
%        and S. Behnke$^{2}$
%%        S. Spencer$^2$\thanks{Chairman Siggraph Publications Board}
%        \\
%% For Computer Graphics Forum: Please use the abbreviation of your first name.
%         $^1$TU Darmstadt \& Fraunhofer IGD, Germany\\
%         $^2$Institut f{\"u}r ComputerGraphik \& Wissensvisualisierung, TU Graz, Austria
%%        $^2$ Another Department to illustrate the use in papers from authors
%%             with different affiliations
%       }

% ------------------------------------------------------------------------

% if the Editors-in-Chief have given you the data, you may uncomment
% the following five lines and insert it here
%
% \volume{27}   % the volume in which the issue will be published;
% \issue{1}     % the issue number of the publication
% \pStartPage{1}      % set starting page


%-------------------------------------------------------------------------
\begin{document}

% \teaser{
%  \includegraphics[width=\linewidth]{eg_new}
%  \centering
%   \caption{New EG Logo}
% \label{fig:teaser}
% }

\maketitle

\begin{abstract}
   This is where we put our abstract.

\begin{classification} % according to http://www.acm.org/class/1998/
\CCScat{Computer Graphics}{I.3.7}{Three-Dimensional Graphics and Realism}{Raytracing}
\end{classification}

\end{abstract}





%-------------------------------------------------------------------------
\section{Introduction}

This is where we write our intro.

%-------------------------------------------------------------------------
\section{Related work}

Our work builds on a long history of research in ray tracing. Below we offer a painfully brief summary of that work, with a particular focus on interactive ray tracing, frameless rendering, and recent applications of frameless rendering.

%-------------------------------------------------------------------------
\subsection{Ray tracing}

Ray tracing references. We should have Turner's paper, one or two other key papers, and cite a textbook on ray tracing. I would suggest Peter Shirley's textbook.

We should also have one or two paragraphs on interactive raytracing. E.g. the RenderCache, a few others that we cited in the AFR paper, and some more recent ones that are cited in the latest edition of Shirley's textbook. We should also mention and cite OptiX.

%-------------------------------------------------------------------------
\subsection{Frameless rendering}

About frameless rendering, adaptive frameless rendering, and recent VR applications of adaptive frameless rendering. Include the system diagram of old AFR from your thesis/the old paper. Figure~\ref{fig:firstExample}.

%------------------------------------------------------------------------
\section{Contributions}

Here we should describe the shortcomings of the previous AFR implementation, and our goals with this work: a unified implementation of AFR (sampling and reconstruction), on a parallel rendering platform.

%-------------------------------------------------------------------------
\section{Implementing AFR on Optix}

Here we describe the details of our implementation. You need to describe the overall structure with your talk diagram noting that reconstruction affects the sampling feedback loop, and then focus in particular on how you manage parallelism.

\noindent Long captions should be set as in Figure~\ref{fig:ex1} or
Figure~\ref{fig:ex3}.

\begin{figure}[htb]
   % an empty figure just consisting of the caption lines
   \caption{\label{fig:ex1}
     'Empty' figure only to serve as an example of long caption requiring 
     more than one line. It is not typed centered but aligned on both sides.}
\end{figure}

\noindent
Figures which need the full textwidth can be typeset as Figure~\ref{fig:ex3}.

%-------------------------------------------------------------------------
\subsection{Sampling controller}

Stuff about the CPU controller


%-------------------------------------------------------------------------
\subsection{Parallel sampler}

Where we describe the GPU sampler, especially managing its parallelism. Describe the deep buffer here, and its locking.

%-------------------------------------------------------------------------
\subsection{Reconstructor}

The GPU reconstructor, and the details of its function and any locking concerns.

%%%
%%% Figure 1
%%%
\begin{figure}[htb]
  \centering
  % the following command controls the width of the embedded PS file
  % (relative to the width of the current column)
  \includegraphics[width=.8\linewidth]{sampleFig}
  % replacing the above command with the one below will explicitly set
  % the bounding box of the PS figure to the rectangle (xl,yl),(xh,yh).
  % It will also prevent LaTeX from reading the PS file to determine
  % the bounding box (i.e., it will speed up the compilation process)
  % \includegraphics[width=.95\linewidth, bb=39 696 126 756]{sampleFig}
  %
  \parbox[t]{.9\columnwidth}{\relax
           For all figures please keep in mind that you \textbf{must not}
           use images with transparent background! 
           }
  %
  \caption{\label{fig:firstExample}
           Here is a sample figure.}
\end{figure}

%-------------------------------------------------------------------------
\section{Results}
Here we show some images produced by our algorithm and some timing results. We describe the hardware we used to produce it.

Figure~\ref{fig:ex3}.

%------------------------------------------------------------------------
\section{Future work}

Your stuff from the thesis/presentation. When we discuss controller on the GPU, we should cite the paper about quadtree on the GPU (https://goo.gl/FlXmqt).

%------------------------------------------------------------------------
\section{Conclusion}

Here we sum up things.

%-------------------------------------------------------------------------
\section{Acknowledgements}

Where we acknowledge folks.

%-------------------------------------------------------------------------

%\bibliographystyle{eg-alpha}
\bibliographystyle{eg-alpha-doi}

\bibliography{egbibsample}


\newpage


\begin{figure*}[tbp]
  \centering
  \mbox{} \hfill
  % the following command controls the width of the embedded PS file
  % (relative to the width of the current column)
  \includegraphics[width=.3\linewidth]{sampleFig}
  % replacing the above command with the one below will explicitly set
  % the bounding box of the PS figure to the rectangle (xl,yl),(xh,yh).
  % It will also prevent LaTeX from reading the PS file to determine
  % the bounding box (i.e., it will speed up the compilation process)
  % \includegraphics[width=.3\linewidth, bb=39 696 126 756]{sampleFig}
  \hfill
  \includegraphics[width=.3\linewidth]{sampleFig}
  \hfill \mbox{}
  \caption{\label{fig:ex3}%
           For publications with color tables (i.e., publications not offering
           color throughout the paper) please \textbf{observe}: 
           for the printed version -- and ONLY for the printed
           version -- color figures have to be placed in the last page.
           \newline
           For the electronic version, which will be converted to PDF before
           making it available electronically, the color images should be
           embedded within the document. Optionally, other multimedia
           material may be attached to the electronic version. }
\end{figure*}

\end{document}
